\documentclass{report}

\author{Sam Huddart}
\title{Tycho 7 Specifcation}
\date {17/03/2021}

\begin{document}
\maketitle

\section {Design}
\subsection{Requirements}

The original purpose of the Tycho binary format was a transmission format, aiming to be small in size while still being
self discribing and offing types that respected the language that I worked in. Other binary formats such as BSON and
MessagePack were unable to fullfill this.

However a new binary format is required for my astra database project, and small transmission siszes, while enticing are not
the main requirement anymore. It makes sense for me to redesign the tycho binary format, to achive theese requirements:

\begin{description}
    \item[-] Simple/Low proccessing to read.
    \item[-] Contain all types of Rust/Serde data model
    \item[-] Support UUIDs
    \item[-] Partial Traversal for large amounts of data.
    \item[-] Small enough for viable tranmission.
\end{description}

\section{Specifcation}
\subsection{Types}
Tycho comes with two types of data, \textbf{Values} and \textbf{Elements}. Values are primative, terminating data such as
strings, booleans, numbers and byte arrays. Elements are non-terminating data and have the ability to contain other
values and elements.

\subsubsection{Values}
Values contain two components; An \textbf{Ident} "prefix" and their respective data. A value's prefix does not need to be
adjacent to value, in a few cases a parent element can define a the Value Ident and the inner values should be read
accordingly. Value's Idents can be one or two bytes long.

\begin{center}
\begin{tabular}{| c | c | c | c | c |}
\textbf{Name} & \textbf{Rust Type} & \textbf{Ident} & \textbf{Data} & \textbf{Description} \\

Null & N/A & 0x00 & No Data & A null type, used to signify no type or no length. \\

Boolean & bool & 0x01 & 1 byte & A boolean type, where the single data byte is 0x00 (false) or 0x01 (true) \\


\end{tabular}
\end{center}

\end{document}