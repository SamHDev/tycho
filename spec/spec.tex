\documentclass{report}

\usepackage{array}
\usepackage{changepage}

\author{Sam Huddart}
\title{Tycho 7 Specifcation}
\date {17/03/2021}

\begin{document}
\maketitle

\section {Design}
\subsection{Requirements}

The original purpose of the Tycho binary format was a transmission format, aiming to be small in size while still being
self discribing and offing types that respected the language that I worked in. Other binary formats such as BSON and
MessagePack were unable to fullfill this.

However a new binary format is required for my astra database project, and small transmission siszes, while enticing are not
the main requirement anymore. It makes sense for me to redesign the tycho binary format, to achive theese requirements:

\begin{description}
    \item[-] Simple/Low proccessing to read.
    \item[-] Contain all types of Rust/Serde data model
    \item[-] Support UUIDs
    \item[-] Partial Traversal for large amounts of data.
    \item[-] Small enough for viable tranmission.
    \item[-] Compression support
    \item[-] Support for digitally signed data.
\end{description}

\section{Specifcation}
\subsection{Lengths}
Inorder to keep byte size down, lengths or arrays or prefixed byte sizes use a variable length number to represent its
respective size.
1
Tycho's Variable Length Numbers are defined as such:

\begin{description}
    \item[-] A number is formed of 8 bit bytes
    \item[-] The 7 least signifacnt bits of an octet represent the number data.
    \item[-] The most signifcant bit, is a logical bit representing if there is another byte.
    \item[-] The number is represented in little endian format.
    \item[-] The number is a 32-bit unsigned number \(5 encoded bytes max\)
\end{description}

Tycho's Variable length encoding is very simmilar to ProtoBuf's uint32.

\subsection{Types}
Tycho comes with two types of data, \textbf{Values} and \textbf{Elements}. Values are primative, terminating data such as
strings, booleans, numbers and byte arrays. Elements are non-terminating data and have the ability to contain other
values and elements.

\subsection{Values}
 Values contain two components; An \textbf{Ident} "prefix" and their respective payload. A value's prefix does not need
 to be adjacent to value, in a few cases a parent element can define a the Value Ident and the inner values should
 be read accordingly. Value's Idents can be one or two bytes long.
 Please allow for any length of ident within your implemenation buy using a byte array rather than reading an u16.

 The payload of a value varies in size should be read in full when parsing partialy.

\newpage
\subsubsection{Base Values}
\vspace{5mm}
\begin{adjustwidth}{-1.5cm}{}
\begin{tabular}{| m{5.5em} | m{4.5em} | m{5em} | m{7em} | m{12em} |}
\hline
\begin{center} \textbf{Name} \end{center} &
 \begin{center} \textbf{Rust} \end{center} &
  \begin{center} \textbf{Ident} \end{center} &
   \begin{center} \textbf{Payload} \end{center} &
    \begin{center} \textbf{Description} \end{center} \\
\hline
Null & N/A & 0x00 & No Data & A null type, used to signify no type or no length. \\
\hline
Boolean & bool & 0x01 & 1 byte & A boolean type, where the single data byte is 0x00 (false) or 0x01 (true) \\
\hline
String & String & 0x02 & length [...bytes] & A UTF-8 String with a given byte length \\
\hline
Char & char & 0x03 & 1-6 bytes & A UTF-8 encoded char with no terminator or length. \\
\hline
Number & ... & 0x04 ... & 1-32 bytes & A number with a given prefix, defined below.  \\
\hline
Bytes & Vec u8 & 0x05 & length [..bytes] & An array of bytes with a given length. \\
\hline
UUID & uuid::Uuid & 0x06 & 16 bytes & A 128-bit Uuid in big-endian  \\
\hline
\end{tabular}
\end{adjustwidth}

\paragraph{Numerical Values}
\begin{adjustwidth}{-1.5cm}{}
\begin{tabular}{| m{5.5em} | m{4.5em} | m{5em} | m{7em} | m{12em} |}
\hline
\begin{center} \textbf{Name} \end{center} &
 \begin{center} \textbf{Rust} \end{center} &
  \begin{center} \textbf{Ident} \end{center} &
   \begin{center} \textbf{Payload} \end{center} &
    \begin{center} \textbf{Description} \end{center} \\
\hline
Bit & N/A & 0x04 0x00 & 1 byte & A single unsigned bit. 0x00 or 0x01 matching 0b0 0b1 respectively.
 In most situations Boolean should be used rather than Bit. \\
\hline
Unsigned8 & u8 & 0x04 0x01 & 1 byte & A unsigned 8 bit number or single octet byte. \\
\hline
Unsigned16 & u16 & 0x04 0x02 & 2 bytes & A big-endian encoded unsigned 16 bit number. \\
\hline
Unsigned32 & u32 & 0x04 0x03 & 4 bytes & A big-endian encoded unsigned 32 bit number. \\
\hline
Unsigned64 & u64 & 0x04 0x04 & 8 bytes & A big-endian encoded unsigned 64 bit number. \\
\hline
Unsigned128 & u128 & 0x04 0x05 & 16 bytes & A big-endian encoded unsigned 128 bit number. \\
\hline
Signed8 & i8 & 0x04 0x11 & 1 byte & A two's complement signed 8bit number. \\
\hline
Signed16 & i16 & 0x04 0x12 & 2 bytes & A big-endian encoded two's complement signed 16 bit number. \\
\hline
Signed32 & i32 & 0x04 0x13 & 4 bytes & A big-endian encoded two's complement signed 32 bit number. \\
\hline
Signed64 & i64 & 0x04 0x14 & 8 bytes & A big-endian encoded two's complement signed 64 bit number. \\
\hline
Signed128 & i128 & 0x04 0x15 & 16 bytes & A big-endian encoded two's complement signed 128 bit number. \\
\hline
Float32 & f32 & 0x04 0x23 & 4 bytes & A IEEE 754 32 bit floating point number. \\
\hline
Float64 & f64 & 0x04 0x24 & 8 bytes & A IEEE 754 64 bit floating point number. \\
\hline
\end{tabular}
\end{adjustwidth}

\subsection{Elements}
An Element is a non-terminating potential container of elements or values. While elements do have Idents like values,
they must be adjacent to the payload. Hence they are considered \textbf{Prefixes} rather then Idents.
Elements have muliple types, each contain diffrent types of payload.


\subsubsection{Data Elements}
Data elements contain at most one inner element depending upon its prefix or payload.

\begin{adjustwidth}{-1.5cm}{}
\begin{tabular}{| m{5.5em} | m{4.5em} | m{5em} | m{7em} | m{12em} |}
\hline
\begin{center} \textbf{Name} \end{center} &
 \begin{center} \textbf{Rust} \end{center} &
  \begin{center} \textbf{Ident} \end{center} &
   \begin{center} \textbf{Payload} \end{center} &
    \begin{center} \textbf{Description} \end{center} \\
\hline

Unit & () & 0x00 & No Data & A Unit/Null type containg no data. \\
Value & ... & 0x01 & Ident Value & A value containing values defined above. \\
None & \tiny Option::None & 0x02 & No Data & A Optional element, where some is false, with no inner element. \\
Some & \tiny Option::Some(T) & 0x03 & Element & A Optional element, where some is true, with a inner element. \\


\end{tabular}
\end{adjustwidth}


\end{document}